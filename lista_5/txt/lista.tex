% !TeX spellcheck = pt_BR

%%%%%%%%%%%%%%%%%%%%%%%%%%%%%%%%%%%%%%%%%%%%%%%
% Modelo adaptado do template original de
% Ted Pavlic (http://www.tedpavlic.com)
% Todos os créditos a ele.
%
% Na versão atual, o que foi modificado
% do original:
% Ajusta a numeração das questões e
% passa para português.
% Além de separar as configurações
% em um arquivo .cls separado.
%
% Crédito ao Roberto por ter feito
% a maior parte do trabalho de passar
% para o português e fazer outros
% ajustes para a versão atual deste template.
%%%%%%%%%%%%%%%%%%%%%%%%%%%%%%%%%%%%%%%%%%%%%%%


%----------------------------------------------------------------------------------------
%	PACKAGES E OUTRAS CONFIGURAÇÕES
%----------------------------------------------------------------------------------------

\documentclass{homeworkclass}

\usepackage{animate}


\usepackage{myMacros}


\hmwkTitle{Lista\ de\ Exercícios \#5}
\hmwkDueDate{Segunda,\ 07\ de\ Julho,\ 2019}
\hmwkClass{Elementos de Processamento de 	Sinais}
\hmwkClassTime{Segundas e Quartas: 08:00--10:00}
\hmwkClassInstructor{Prof.\ Sergio Lima Netto}
\hmwkAuthorName{Vinicius Mesquita de Pinho}
\hmwkAuthorShortName{Vinicius Mesquita}

\begin{document}

\maketitle

%----------------------------------------------------------------------------------------
%	SUMÁRIO
%----------------------------------------------------------------------------------------

%\setcounter{tocdepth}{1} % Uncomment this line if you don't want subsections listed in the ToC

\clearpage
\newpage
%\tableofcontents
%\newpage

%----------------------------------------------------------------------------------------
%	QUESTÃO 1
%----------------------------------------------------------------------------------------

% To have just one problem per page, simply put a \clearpage after each problem



\begin{homeworkProblem}

\end{homeworkProblem}

\end{document}